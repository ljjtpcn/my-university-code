\documentclass[a4paper, 12pt]{article}
\usepackage[UTF8]{ctex}

%这是文章的标题
\title{LaTex 常用模板}
%这是文章的作者
\author{Kevin}
%这是文章的时间
%如果没有这行将显示当前时间
%如果不想显示时间则使用 \date{}
\date{2008/10/12}
%以上部分叫做"导言区",下面才开始写正文
\begin{document}

%先插入标题
\maketitle     %主要的作用是用于生成标题的作用 content contain \title \author \date
%再插入目录
\tableofcontents   %主要的作用适用于生成目录的作用
\section{LaTex 简介}
LaTex是一个宏包,目的是使作者能够利用一个
预先定义好的专业页面设置,
从而得以高质量的排版和打印他们的作品.
%第二段使用黑体,上面的一个空行表示另起一段
\CJKfamily{hei}LaTex 将空格和制表符视为相同的距离.
多个连续的空白字符 等同为一个空白字符
\section{LaTex源文件}
%在第二段我们使用隶书
\CJKfamily{li}LaTex 源文件格式为普通的ASCII文件,
你可以使用任何文本编辑器来创建.
LaTex源文件不仅包括你要排版的文本, 还包括LaTex
所能识别的,如何排版这些文本的命令.
\section{结论}
%在结论部分我们使用仿宋体
\CJKfamily{fs}LaTeX, 我看行! $a_{i, j}$

$\overbrace{1+2+\cdots+100}$


\end{document}
\documentclass{article}
\usepackage{CJK}
 

